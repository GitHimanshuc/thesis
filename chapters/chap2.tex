\chapter{Introduction to critical Scalar collapse}

In our case we were not so much interested in the behavior of the field close to criticality. What we wanted to understand was how much the scalar field moves during the black hole formation so we can relate it to the swampland distance bound.

\begin{equation}M \simeq k\left(p-p_{*}\right)^{\gamma}\end{equation}
\begin{equation}\phi_{*}(r, t)=\phi_{*}\left(e^{\Delta} r, e^{\Delta} t\right)\end{equation}


\begin{equation}
    -\psi_{, t t}+\psi_{, x x}+\frac{2}{r}\left(-r_{, t} \psi_{, t}+r_{, x} \psi_{, x}\right)=0
\end{equation}

\begin{equation}
    -r_{, t t}+r_{, x x}-e^{-2 \sigma} \cdot \frac{2 m}{r^{2}}=0
\end{equation}

\begin{equation}
    -\sigma_{, t t}+\sigma_{, x x}-e^{-2 \sigma} \cdot \frac{2 m}{r^{3}}+4 \pi\left(\psi_{, t}^{2}-\psi_{, x}^{2}\right)=0
\end{equation}

\begin{equation}m_{, t}=4 \pi r^{2} \cdot e^{2 \sigma}\left[-\frac{1}{2} r_{, t}\left(\psi_{, t}^{2}+\psi_{, x}^{2}\right)+r_{, x} \psi_{, t} \psi_{, x}\right]\end{equation}


\section{Boundary and Initial conditions} \label{sec:boundary_and_initial_conditions}

There are four variables in our equations $r,m,\sigma,\psi$. At time $t=0$ we only have the value of $\psi$ for all the values of $x$, i.e. we have profile of the initial scalar field configuration. Before we can evolve the system in time we need the profile of other three variable at $t=0$. That being said we are not free to choose any profile for $r,m,\sigma$ because even at $t=0$ they have to satisfy the Einstein equations to represent a physical system.


\begin{equation}
    r_{, x x}=e^{-2 \sigma} \cdot \frac{2 m}{r^{2}}
\end{equation}\label{eqn:r_chap3}


\begin{equation}
    \sigma_{, x}= -2 \pi \cdot  \frac{\psi_{, x}^{2} \cdot r}{r_{,x}}- e^{-2 \sigma} \cdot \frac{ m}{r^{2}r_{, x}}
\end{equation}\label{eqn:sigma_chap3}

\begin{equation}m_{, x}=4 \pi r^{2} \cdot e^{2 \sigma}\left[\frac{1}{2} r_{, x} \cdot \psi_{, x}^{2} \right]\end{equation}


with boundary conditions,

\begin{eqnarray}
    r(0 ,0) = 0 \implies r^0_0 = 0 \\
    r_{,x}(0 ,0) = 1 \implies r^0_{0,x} = 1 \\
    \sigma(0 ,0) = 1 \implies \sigma^0_0 = 0 \\
    m(0 ,0) = 1 \implies m^0_0 = 0
\end{eqnarray}

Notice that we need four boundary conditions in total and we have them.
